\documentclass[twocolumn]{article}

\newcommand{\authorname}{Austin J. Maddison}
\newcommand{\course}{Discrete Simulation}
\newcommand{\courseid}{ICMA393}
\newcommand{\docname}{HW1}
\newcommand{\titletext}{\course: \docname}

%


%\usepackage{fontspec}	
\usepackage[no-math]{fontspec}	
\setmainfont{HelveticaNowText}
\newfontface\hh{HelveticaNowText-ExtraBold}
\newfontface\lt{HelveticaNowText-Light}      
\newfontface\xx{HelveticaNowText-ExtraLight} 
\newfontface\mm{HelveticaNowText Medium}

\newfontfamily{\displayfont}{HelveticaNowDisplay}
\newfontface\dslt{HelveticaNowDisplay-Light}
\newfontface\dsmm{HelveticaNowDisplay-Medium}
\newfontface\dsbd{HelveticaNowDisplay-Bold}

\newfontfamily{\microfont}{HelveticaNowMicro}
\newfontface\mclt{HelveticaNowMicro-Light}
\newfontface\mcmm{HelveticaNowMicro-Medium}
\newfontface\mcbd{HelveticaNowMicro-Bold}

\setmonofont{SFMono}

\usepackage[lining]{FiraSans}
\usepackage[fakebold]{firamath-otf}
\usepackage{unicode-math}
\setmathfont{FiraMath}
\renewcommand*\oldstylenums[1]{{\firaoldstyle #1}}




\usepackage{microtype}   % Improves text appearance with microtypography
\usepackage{amsmath}     % For better math support
\usepackage{graphicx}    % For including graphics
\usepackage{lipsum}      % For placeholder text
\usepackage{enumitem}
\usepackage{xcolor}
\usepackage{svg}
\usepackage{svg-extract}
\usepackage{caption}
\usepackage{float}


\usepackage[a4paper, margin=0.8in, columnsep=20pt]{geometry}

\captionsetup{font=small}
\definecolor{gray}{rgb}{0.55, 0.55, 0.55}
\setlength{\columnsep}{20pt}  % Space between columns

% Headers and Footers
\usepackage{fancyhdr}
\pagestyle{fancy}
\fancyhf{}

% First Page
\fancypagestyle{plain}{
\fancyfoot[R]{\small \thepage} 
\fancyhead[L]{}
\fancyhead[R]{}
}

% Custom header
\fancyhead[L]{\scriptsize \MakeUppercase{ \microfont \courseid~\course}}
\fancyhead[R]{\scriptsize \MakeUppercase{ \microfont \docname}}
\renewcommand{\headrulewidth}{0pt}

% Custom footer
%\fancyfoot[L]{\small Title, Date}
\fancyfoot[R]{\small \thepage}

% Line spacing
\usepackage{setspace}
\setstretch{1.15}  % Slightly more space between lines

%\setlength{\mathindent}{0pt} % This removes the indentation for equations

% Section formatting
\usepackage{titlesec}
\titleformat{\section}[block]{\large\dsbd}{\thesection.}{1em}{}
\titleformat{\subsection}[block]{\normalsize\mm}{\thesubsection.}{1em}{}

% Bibliography style
\usepackage[numbers,sort&compress]{natbib} % For numbered citations

% Hyperlinks
\usepackage{hyperref}
\hypersetup{
    colorlinks=false,
    linkcolor=blue,
    citecolor=blue,
    urlcolor=blue,
    pdftitle={Research Paper Title},
    pdfauthor={Author's Name},
}

\usepackage{listings}
\lstset{
  language=Python,                     % Use Python language syntax
  basicstyle=\ttfamily\footnotesize,           % Use modern monospace font for code
  keywordstyle=\bfseries\color{black},   % Bold and blue keywords
  stringstyle=\color{black},              % Strings in red
  commentstyle=\color{gray},            % Comments in gray
  showstringspaces=false,               % Don't show spaces in strings
  breaklines=true,                      % Break long lines
  tabsize=4,                            % Set tab size to 4 spaces
}

\begin{document}
%\fontsize{10}{10.5}\selectfont % Set font size to 12pt with a baseline of 14pt


% Title 
\title{\displayfont \strong{\titletext}}
\author{\small \authorname \\ \small Mahidol University International College}
\date{\small \today}


%
\twocolumn[{
  \maketitle
%  \begin{abstract}
%    \noindent Your abstract goes here. This is a brief summary of your paper.
%  \end{abstract}
%  \vspace{1cm}
}]




\section{Birthday Problem}\label{p1}

The simulations ran were set to $seed=27$ \\and $N=100000$.

\begin{enumerate}[label=\alph*)]
    
\item \begin{align*}
p_{10} &= 0.11686 \\
CI &= [0.11685, 0.11885]
\end{align*}
    
\item \begin{align*}
p_{20} &=  0.41126 \\
CI &= [0.41125, 0.41430]
\end{align*}

\item \begin{align*}
p_{30} &= 0.70595 \\
CI &= [0.70594, 0.70877]
\end{align*}


\item 
Found n that satisfies the condition $p_n \ge 0.5$  from extracting the smallest difference from varying $n$'s of $p_n$ and the function of $p=0.5$.

	
\begin{figure}[H]
    \centering
    \includesvg[width=0.5\textwidth]{p1_d.svg} % Adjust the width as needed
    \caption{Closest point to $p=0.5$ was $p_{23}=0.54$ which satisfies the condition.}
\end{figure}

\end{enumerate}
\pagebreak

\section{Alice and Bob Play a Game}
The strategy I let Alice have is the following...
\\\\
\noindent
\textbf{Case 1: Found Exclusive Output}\\
She presses the button recording $n$ output values $x_1, x_2, ... x_n$.
As she records each output she checks whether $x_i$ is exclusive to one of the buttons output range. If $x_i$ is exclusive she returns the corresponding button as the answer.
\\\\
\textbf{Case 2: No Exclusive Output}\\
If she presses the button $n$ times with no exclusive output appearing. She calculates $\bar{x}$ and finds the minimum difference between the mean output range of the 2 buttons. She returns the corresponding button that gets the minimum distance.
\\\\
\textbf{Psuedocode}\\
Although the source code differs because loops are removed for speed, the idea is the same.
%\vspace{-4pt}
\begin{lstlisting}
xs = []
for i in range(0, n):	
	x_n = button_unknown.get_next_value()
	xs.append(x_n) 	
	
	# Case 1: Found exclusive output.
	if(x_n == 1)
		return 1  # it is button 1
	if(x_n == 100)
		return 2  # it is button 2
		
# Case 2: No exclusive output. Evaluate the minimum distance of means.
x_mean = sum(xs) / n
button_1_mean = (1 + 99)/2
button_2_mean = (2 + 100)/2

return argmin([abs(x_mean - button_1_mean), abs(x_mean - button_2_mean)]) + 1
\end{lstlisting}

\subsection*{Find $n$ such that Alice is correct $\ge 0.99$}
Similar to problem \ref{p1} \nameref{p1}, I sampled Alice's strategy over some ``reasonable'' range of $n$ $[0, 500]$ and extract the n that results in probability $p$ that Alice is correct is atleast $0.99$. I set the $seed=27$ and  ran $400000$ trials for all $n$'s.


\begin{figure}[H]
    \centering
    \includesvg[width=0.5\textwidth]{p2_1.svg} % Adjust the width as needed
    \caption{Closest point to $p=0.99$ was $p_{384}=0.990$ which satisfies the condition.}
    \label{fig:Figure2}
\end{figure}

\begin{figure}[H]
    \centering
    \includesvg[width=0.5\textwidth]{p2_2.svg} % Adjust the width as needed
    \caption{y-lim[0.8, 1.0] of Figure~\ref{fig:Figure2}.}
\end{figure}


\section{Practice with Uniform and Geometric Distributions}


\section*{Source Code}
\href{https://github.com/AustinMaddison/discrete-simulation/tree/main/src/hw1}{https://github.com/AustinMaddison/discrete-simulation/tree/main/src/hw1}


% References
%\bibliographystyle{unsrt}
%\bibliography{references}


\end{document}
